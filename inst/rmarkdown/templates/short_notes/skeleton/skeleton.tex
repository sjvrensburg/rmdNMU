\documentclass[
  14pt,
  a4paper,
]{extarticle}

% Load basic packages first
\usepackage{fontspec}
\usepackage{amssymb}
\usepackage{unicode-math}
\usepackage{microtype}
\usepackage{graphicx}

% Image handling settings
\makeatletter
\def\maxwidth{\ifdim\Gin@nat@width>\linewidth\linewidth\else\Gin@nat@width\fi}
\def\maxheight{\ifdim\Gin@nat@height>\textheight\textheight\else\Gin@nat@height\fi}
\makeatother
% Scale images if necessary, so that they will not overflow the page
% margins by default, and it is still possible to overwrite the defaults
% using explicit options in \includegraphics[width, height, ...]{}
\setkeys{Gin}{width=\maxwidth,height=\maxheight,keepaspectratio}

\usepackage{fancyvrb}
\usepackage{xcolor}
\usepackage{bookmark}
\usepackage{xurl}
\usepackage{geometry}
\usepackage{setspace}
\usepackage{parskip}

\usepackage{tabularx}
\usepackage{array}
\usepackage{longtable}
\usepackage{booktabs}
\usepackage{multirow}

\usepackage{calc}  % For length calculations
\makeatletter
\def\real#1{\strip@pt\dimexpr #1pt\relax}
\makeatother

% Define column types for tables
\newcolumntype{L}[1]{>{\raggedright\arraybackslash}p{#1}}
\newcolumntype{C}[1]{>{\centering\arraybackslash}p{#1}}
\newcolumntype{R}[1]{>{\raggedleft\arraybackslash}p{#1}}

% Set default table formatting
\setlength{\tabcolsep}{6pt}
\renewcommand{\arraystretch}{1.2}

% Basic setup
\SetProtrusion
  [ name     = microtype-default,
    load     = microtype-default ]
  { }
  { }

% Font configuration
\defaultfontfeatures{
    Ligatures=TeX,
    UprightFeatures={Weight=400},  % Regular weight for normal text
    BoldFeatures={Weight=700},     % Bold weight
    ItalicFeatures={Weight=400},   % Regular weight for italics
    BoldItalicFeatures={Weight=700} % Bold weight for bold italics
}

\setmainfont{Nunito Sans}[
    Extension = .ttf,
    Path = /home/stefan/R/x86_64-pc-linux-gnu-library/4.3/rmdNMU/fonts/nunito-sans/,
    UprightFont = NunitoSans-Regular,
    ItalicFont = NunitoSans-Italic,
    BoldFont = NunitoSans-Regular,   % Using Regular font with weight variation
    BoldItalicFont = NunitoSans-Italic,  % Using Italic font with weight variation
    % Font Features section
    UprightFeatures = {
        Weight=450,  % Slightly heavier than regular for better readability
        Scale=1.0
    },
    BoldFeatures = {
        Weight=700  % Bold
    },
    ItalicFeatures = {
        Weight=450  % Matching upright weight
    },
    BoldItalicFeatures = {
        Weight=700  % Bold italic
    }
]

\setmathfont{KpMath-Sans}[
    Scale=MatchUppercase,
    StylisticSet=1
]

% For bold math symbols
\AtBeginDocument{
    \let\origbold\mathbf
    \renewcommand{\mathbf}[1]{\symbfit{#1}}
}

% Code highlighting
\usepackage{color}
\usepackage{fancyvrb}
\newcommand{\VerbBar}{|}
\newcommand{\VERB}{\Verb[commandchars=\\\{\}]}
\DefineVerbatimEnvironment{Highlighting}{Verbatim}{commandchars=\\\{\}}
% Add ',fontsize=\small' for more characters per line
\usepackage{framed}
\definecolor{shadecolor}{RGB}{248,248,248}
\newenvironment{Shaded}{\begin{snugshade}}{\end{snugshade}}
\newcommand{\AlertTok}[1]{\textcolor[rgb]{0.94,0.16,0.16}{#1}}
\newcommand{\AnnotationTok}[1]{\textcolor[rgb]{0.56,0.35,0.01}{\textbf{\textit{#1}}}}
\newcommand{\AttributeTok}[1]{\textcolor[rgb]{0.13,0.29,0.53}{#1}}
\newcommand{\BaseNTok}[1]{\textcolor[rgb]{0.00,0.00,0.81}{#1}}
\newcommand{\BuiltInTok}[1]{#1}
\newcommand{\CharTok}[1]{\textcolor[rgb]{0.31,0.60,0.02}{#1}}
\newcommand{\CommentTok}[1]{\textcolor[rgb]{0.56,0.35,0.01}{\textit{#1}}}
\newcommand{\CommentVarTok}[1]{\textcolor[rgb]{0.56,0.35,0.01}{\textbf{\textit{#1}}}}
\newcommand{\ConstantTok}[1]{\textcolor[rgb]{0.56,0.35,0.01}{#1}}
\newcommand{\ControlFlowTok}[1]{\textcolor[rgb]{0.13,0.29,0.53}{\textbf{#1}}}
\newcommand{\DataTypeTok}[1]{\textcolor[rgb]{0.13,0.29,0.53}{#1}}
\newcommand{\DecValTok}[1]{\textcolor[rgb]{0.00,0.00,0.81}{#1}}
\newcommand{\DocumentationTok}[1]{\textcolor[rgb]{0.56,0.35,0.01}{\textbf{\textit{#1}}}}
\newcommand{\ErrorTok}[1]{\textcolor[rgb]{0.64,0.00,0.00}{\textbf{#1}}}
\newcommand{\ExtensionTok}[1]{#1}
\newcommand{\FloatTok}[1]{\textcolor[rgb]{0.00,0.00,0.81}{#1}}
\newcommand{\FunctionTok}[1]{\textcolor[rgb]{0.13,0.29,0.53}{\textbf{#1}}}
\newcommand{\ImportTok}[1]{#1}
\newcommand{\InformationTok}[1]{\textcolor[rgb]{0.56,0.35,0.01}{\textbf{\textit{#1}}}}
\newcommand{\KeywordTok}[1]{\textcolor[rgb]{0.13,0.29,0.53}{\textbf{#1}}}
\newcommand{\NormalTok}[1]{#1}
\newcommand{\OperatorTok}[1]{\textcolor[rgb]{0.81,0.36,0.00}{\textbf{#1}}}
\newcommand{\OtherTok}[1]{\textcolor[rgb]{0.56,0.35,0.01}{#1}}
\newcommand{\PreprocessorTok}[1]{\textcolor[rgb]{0.56,0.35,0.01}{\textit{#1}}}
\newcommand{\RegionMarkerTok}[1]{#1}
\newcommand{\SpecialCharTok}[1]{\textcolor[rgb]{0.81,0.36,0.00}{\textbf{#1}}}
\newcommand{\SpecialStringTok}[1]{\textcolor[rgb]{0.31,0.60,0.02}{#1}}
\newcommand{\StringTok}[1]{\textcolor[rgb]{0.31,0.60,0.02}{#1}}
\newcommand{\VariableTok}[1]{\textcolor[rgb]{0.00,0.00,0.00}{#1}}
\newcommand{\VerbatimStringTok}[1]{\textcolor[rgb]{0.31,0.60,0.02}{#1}}
\newcommand{\WarningTok}[1]{\textcolor[rgb]{0.56,0.35,0.01}{\textbf{\textit{#1}}}}

% List formatting
\providecommand{\tightlist}{%
  \setlength{\itemsep}{0pt}\setlength{\parskip}{0pt}}

% URL styling
\urlstyle{same}

% Load the preamble with custom environments and styles
% Additional required packages (not loaded in template)
\usepackage{titlesec}
\usepackage{amsthm}
\usepackage{thmtools}
\usepackage[most]{tcolorbox}
\usepackage{emoji}
\usepackage{amsfonts}
\usepackage{cleveref}

% Color definitions
\definecolor{nmuprimaryblue}{HTML}{141C2B}
\definecolor{nmusecondaryblue}{HTML}{132E51}
\definecolor{nmuprimaryyellow}{HTML}{FFCC00}
\definecolor{nmusecondaryyellow}{HTML}{F9B22A}
\definecolor{nmugrey}{HTML}{999999}
\definecolor{nmuhumanities}{HTML}{FFB51B}
\definecolor{nmueducation}{HTML}{F14F13}
\definecolor{nmubusiness}{HTML}{6C284F}
\definecolor{nmusciences}{HTML}{006B34}
\definecolor{nmuhealth}{HTML}{82B74A}
\definecolor{nmuengineering}{HTML}{57BCE9}
\definecolor{nmulaw}{HTML}{5E6EBA}
\definecolor{nmuocean}{HTML}{00AFAA}

% Page setup
\geometry{a4paper, margin=2.54cm}

% Section formatting with heavier weights
\titleformat{\section}{\Large\bfseries\addfontfeatures{Weight=800}\color{nmuprimaryblue}}{\thesection}{1em}{}
\titleformat{\subsection}{\large\bfseries\addfontfeatures{Weight=700}\color{nmusecondaryblue}}{\thesubsection}{1em}{}
\titleformat{\subsubsection}{\normalsize\bfseries\addfontfeatures{Weight=700}\color{nmusecondaryblue}}{\thesubsubsection}{1em}{}

% Default text color
\color{nmuprimaryblue}

% Theorem styles with proper weights
\declaretheoremstyle[
   headfont=\color{nmusecondaryblue}\addfontfeatures{Weight=800}\bfseries,
   bodyfont=\itshape,
   spaceabove=10pt,
   spacebelow=10pt,
   headpunct={:}
]{nmustyle}

\declaretheoremstyle[
   headfont=\color{nmusecondaryblue}\addfontfeatures{Weight=800}\bfseries,
   bodyfont=\itshape,
   spaceabove=10pt,
   spacebelow=10pt,
   headpunct={:}
]{nmudefinitionstyle}

\declaretheoremstyle[
   headfont=\color{nmusecondaryblue}\itshape\addfontfeatures{Weight=800}\bfseries,
   bodyfont=\itshape,
   spaceabove=10pt,
   spacebelow=10pt,
   headpunct={--}
]{nmuremarkstyle}


% Section numbering
\setcounter{secnumdepth}{3}

% Theorem environments
\declaretheorem[style=nmustyle,name=Theorem,within=section]{theorem}
\declaretheorem[style=nmustyle,name=Lemma,sibling=theorem]{lemma}
\declaretheorem[style=nmustyle,name=Proposition,sibling=theorem]{proposition}
\declaretheorem[style=nmustyle,name=Corollary,sibling=theorem]{corollary}
\declaretheorem[style=nmudefinitionstyle,name=Definition,numberwithin=section]{definition}
\declaretheorem[style=nmudefinitionstyle,name=Example,sibling=definition]{example}
\declaretheorem[style=nmuremarkstyle,name=Remark,numberwithin=section]{remark}
\declaretheorem[style=nmuremarkstyle,name=Note,sibling=remark]{note}

% Proof environment
\renewenvironment{proof}{%
   \par\noindent{\color{nmuprimaryblue}\addfontfeatures{Weight=700}Proof:}\normalfont}{%
   \hfill$\blacksquare$\par}

% Custom box titles
\makeatletter
\newcommand{\customboxtitle}[2][]{%
   \ifx&#1&%
       #2%
   \else%
       #2\ #1%
   \fi%
}
\makeatother

% Custom boxes
\newtcolorbox{warningbox}[1][]{
   enhanced,
   colback=red!5,
   colframe=red!80!black,
   arc=0mm,
   title={\large\emoji{warning} #1},
   fonttitle=\fontseries{sb}\selectfont,
   left=5mm,right=5mm,top=5mm,bottom=5mm
}

\newtcolorbox{ideabox}[1][]{
   enhanced,
   colback=nmuprimaryyellow!10,
   colframe=nmusecondaryyellow,
   arc=0mm,
   title={\large\emoji{light-bulb}  #1},
   fonttitle=\fontseries{sb}\selectfont,
   left=5mm,right=5mm,top=5mm,bottom=5mm
}

\newtcolorbox{notebox}[1][]{
   enhanced,
   colback=nmuprimaryblue!5,
   colframe=nmuprimaryblue,
   arc=0mm,
   title={\large\emoji{memo}  #1},
   fonttitle=\fontseries{sb}\selectfont,
   left=5mm,right=5mm,top=5mm,bottom=5mm
}

\newtcolorbox{tipbox}[1][]{
   enhanced,
   colback=nmuocean!10,
   colframe=nmuocean,
   arc=0mm,
   title={\large\emoji{thought-balloon} #1},
   fonttitle=\fontseries{sb}\selectfont,
   left=5mm,right=5mm,top=5mm,bottom=5mm
}

\newtcolorbox{importantbox}[1][]{
   enhanced,
   colback=nmusciences!10,
   colframe=nmusciences,
   arc=0mm,
   title={\large\emoji{glowing-star} #1},
   fonttitle=\fontseries{sb}\selectfont,
   left=5mm,right=5mm,top=5mm,bottom=5mm
}

\newtcolorbox{examplebox}[1][]{
   enhanced,
   colback=nmuengineering!10,
   colframe=nmuengineering,
   arc=0mm,
   title={\large\emoji{bar-chart} #1},
   fonttitle=\fontseries{sb}\selectfont,
   left=5mm,right=5mm,top=5mm,bottom=5mm
}

% Cross-referencing setup
\crefname{definition}{Definition}{Definitions}
\crefname{theorem}{Theorem}{Theorems}
\crefname{lemma}{Lemma}{Lemmas}
\crefname{proposition}{Proposition}{Propositions}
\crefname{corollary}{Corollary}{Corollaries}
\crefname{example}{Example}{Examples}
\crefname{remark}{Remark}{Remarks}
\crefname{note}{Note}{Notes}

% Hyperref setup
\hypersetup{
   colorlinks=true,
   linkcolor=nmuprimaryblue,
   filecolor=nmuprimaryblue,
   urlcolor=nmuprimaryblue,
   citecolor=nmuprimaryblue
}

% PDF metadata
\usepackage{hyperref}
\hypersetup{
  pdftitle={Short Notes Test Document},
  pdfauthor={Your Name},
  hidelinks,
  pdfcreator={LaTeX via pandoc}
}

% Bibliography setup if needed

% Title information
\title{Short Notes Test Document}
\author{Your Name}
\date{2025-02-06}

\begin{document}
\maketitle

\section{\texorpdfstring{\emoji{memo}
\textbf{Introduction}}{ Introduction}}\label{introduction}

This is an example of a \textbf{Short Notes document} using the NMU
template. It demonstrates key features such as structured sections,
example boxes, mathematical notation, and inline code.

\begin{center}\rule{0.5\linewidth}{0.5pt}\end{center}

\section{\texorpdfstring{\emoji{pushpin} \textbf{Basic
Formatting}}{ Basic Formatting}}\label{basic-formatting}

\subsection{\texorpdfstring{\emoji{small-blue-diamond}
\textbf{Lists}}{ Lists}}\label{lists}

\begin{itemize}
\tightlist
\item
  Bullet points are useful for key ideas.\\
\item
  Numbered lists help organize steps:
\end{itemize}

\begin{enumerate}
\def\labelenumi{\arabic{enumi}.}
\tightlist
\item
  First item\\
\item
  Second item\\
\item
  Third item
\end{enumerate}

\subsection{\texorpdfstring{\emoji{small-blue-diamond} \textbf{Text
Formatting}}{ Text Formatting}}\label{text-formatting}

\begin{itemize}
\tightlist
\item
  \textbf{Bold Text}\\
\item
  \emph{Italic Text}\\
\item
  \texttt{Inline\ code} example
\end{itemize}

\begin{quote}
\emoji{light-bulb} \textbf{Tip:} Use \texttt{\#} for headings and
\texttt{*} for italics.
\end{quote}

\begin{center}\rule{0.5\linewidth}{0.5pt}\end{center}

\section{\texorpdfstring{\emoji{open-book} \textbf{Mathematics \&
Equations}}{ Mathematics \& Equations}}\label{mathematics-equations}

Mathematical expressions can be \textbf{inline} like this:\\
``The equation of a line is \(y = mx + c\).''

Or \textbf{displayed} in a block: \[
E = mc^2
\]

You can have aligned equation environments: \(\beta\)

\begin{align*}
f(\boldsymbol{\beta}) &= (\mathbf{y} - \mathbf{X}\boldsymbol{\beta})'\,(\mathbf{y} - \mathbf{X}\boldsymbol{\beta}) \\
f'(\boldsymbol{\beta}) &= \frac{d}{d\boldsymbol{\beta}} 
\end{align*}

\begin{quote}
\emoji{hammer-and-wrench} \textbf{Try it!} Modify the equation to
explore different formatting.
\end{quote}

\begin{center}\rule{0.5\linewidth}{0.5pt}\end{center}

\subsection*{\texorpdfstring{\emoji{trophy} \textbf{Theorem \& Proof
Example}}{ Theorem \& Proof Example}}\label{theorem-proof-example}
\addcontentsline{toc}{subsection}{\emoji{trophy} \textbf{Theorem \&
Proof Example}}

\begin{theorem}[Pythagoras' Theorem]\label{thm:pythagoras}

In a right-angled triangle with sides \(a\) and \(b\), and hypotenuse
\(c\): \[
a^2 + b^2 = c^2
\]

\end{theorem}

\begin{proof}

By applying the properties of similar triangles, we derive: \[
a^2 + b^2 = c^2
\] Thus, the theorem holds.

\end{proof}

\begin{center}\rule{0.5\linewidth}{0.5pt}\end{center}

\section{\texorpdfstring{\emoji{package} \textbf{Custom
Boxes}}{ Custom Boxes}}\label{custom-boxes}

Different types of \textbf{callout boxes} highlight important
information.

\begin{examplebox}[Example]

The Fibonacci sequence starts as follows: \[
0, 1, 1, 2, 3, 5, 8, 13, \dots
\]

\end{examplebox}

\begin{importantbox}[Important]

Understanding how to apply formulas is essential in problem-solving.

\end{importantbox}

\begin{warningbox}[Warning]

Be careful when applying formulas---always check your assumptions!

\end{warningbox}

\begin{quote}
\emoji{check-mark-button} \textbf{Test it!} Try adding your own
\textbf{definitions} or \textbf{examples}.
\end{quote}

\begin{center}\rule{0.5\linewidth}{0.5pt}\end{center}

\section{\texorpdfstring{\emoji{desktop-computer} \textbf{Code Blocks in
R}}{ Code Blocks in R}}\label{code-blocks-in-r}

You can include \textbf{R code} like this:

\begin{Shaded}
\begin{Highlighting}[]
\NormalTok{fib }\OtherTok{\textless{}{-}} \ControlFlowTok{function}\NormalTok{(n) \{}
  \ControlFlowTok{if}\NormalTok{ (n }\SpecialCharTok{\textless{}=} \DecValTok{1}\NormalTok{) }\FunctionTok{return}\NormalTok{(n)}
  \FunctionTok{return}\NormalTok{(}\FunctionTok{fib}\NormalTok{(n}\DecValTok{{-}1}\NormalTok{) }\SpecialCharTok{+} \FunctionTok{fib}\NormalTok{(n}\DecValTok{{-}2}\NormalTok{))}
\NormalTok{\}}

\FunctionTok{fib}\NormalTok{(}\DecValTok{10}\NormalTok{) }\CommentTok{\# Compute the 10th Fibonacci number}
\end{Highlighting}
\end{Shaded}

\begin{verbatim}
## [1] 55
\end{verbatim}

\begin{quote}
\emoji{bullseye} \textbf{Tip:} Modify the function to calculate a
different number.
\end{quote}

\begin{center}\rule{0.5\linewidth}{0.5pt}\end{center}

\section{\texorpdfstring{\emoji{bullseye}
\textbf{Conclusion}}{ Conclusion}}\label{conclusion}

This \textbf{Short Notes template} makes it easy to create
well-structured, professional-looking documents for teaching, learning,
and research. \emoji{rocket}

\begin{quote}
\emoji{mortar-board} \textbf{Now, try editing this document to make it
your own!}
\end{quote}

\begin{center}\rule{0.5\linewidth}{0.5pt}\end{center}

\section{\texorpdfstring{\emoji{pushpin} \textbf{Final
Thoughts}}{ Final Thoughts}}\label{final-thoughts}

This skeleton is \textbf{beginner-friendly} while still showcasing all
key features of the \textbf{Short Notes} template. Let me know if you'd
like any refinements! \emoji{rocket}


\end{document}
